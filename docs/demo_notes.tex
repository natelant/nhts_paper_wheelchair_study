\documentclass[]{article}
\usepackage{lmodern}
\usepackage{amssymb,amsmath}
\usepackage{ifxetex,ifluatex}
\usepackage{fixltx2e} % provides \textsubscript
\ifnum 0\ifxetex 1\fi\ifluatex 1\fi=0 % if pdftex
  \usepackage[T1]{fontenc}
  \usepackage[utf8]{inputenc}
\else % if luatex or xelatex
  \ifxetex
    \usepackage{mathspec}
  \else
    \usepackage{fontspec}
  \fi
  \defaultfontfeatures{Ligatures=TeX,Scale=MatchLowercase}
\fi
% use upquote if available, for straight quotes in verbatim environments
\IfFileExists{upquote.sty}{\usepackage{upquote}}{}
% use microtype if available
\IfFileExists{microtype.sty}{%
\usepackage{microtype}
\UseMicrotypeSet[protrusion]{basicmath} % disable protrusion for tt fonts
}{}
\usepackage[margin=1in]{geometry}
\usepackage{hyperref}
\hypersetup{unicode=true,
            pdftitle={notes},
            pdfauthor={Nate Lant},
            pdfborder={0 0 0},
            breaklinks=true}
\urlstyle{same}  % don't use monospace font for urls
\usepackage{color}
\usepackage{fancyvrb}
\newcommand{\VerbBar}{|}
\newcommand{\VERB}{\Verb[commandchars=\\\{\}]}
\DefineVerbatimEnvironment{Highlighting}{Verbatim}{commandchars=\\\{\}}
% Add ',fontsize=\small' for more characters per line
\usepackage{framed}
\definecolor{shadecolor}{RGB}{248,248,248}
\newenvironment{Shaded}{\begin{snugshade}}{\end{snugshade}}
\newcommand{\AlertTok}[1]{\textcolor[rgb]{0.94,0.16,0.16}{#1}}
\newcommand{\AnnotationTok}[1]{\textcolor[rgb]{0.56,0.35,0.01}{\textbf{\textit{#1}}}}
\newcommand{\AttributeTok}[1]{\textcolor[rgb]{0.77,0.63,0.00}{#1}}
\newcommand{\BaseNTok}[1]{\textcolor[rgb]{0.00,0.00,0.81}{#1}}
\newcommand{\BuiltInTok}[1]{#1}
\newcommand{\CharTok}[1]{\textcolor[rgb]{0.31,0.60,0.02}{#1}}
\newcommand{\CommentTok}[1]{\textcolor[rgb]{0.56,0.35,0.01}{\textit{#1}}}
\newcommand{\CommentVarTok}[1]{\textcolor[rgb]{0.56,0.35,0.01}{\textbf{\textit{#1}}}}
\newcommand{\ConstantTok}[1]{\textcolor[rgb]{0.00,0.00,0.00}{#1}}
\newcommand{\ControlFlowTok}[1]{\textcolor[rgb]{0.13,0.29,0.53}{\textbf{#1}}}
\newcommand{\DataTypeTok}[1]{\textcolor[rgb]{0.13,0.29,0.53}{#1}}
\newcommand{\DecValTok}[1]{\textcolor[rgb]{0.00,0.00,0.81}{#1}}
\newcommand{\DocumentationTok}[1]{\textcolor[rgb]{0.56,0.35,0.01}{\textbf{\textit{#1}}}}
\newcommand{\ErrorTok}[1]{\textcolor[rgb]{0.64,0.00,0.00}{\textbf{#1}}}
\newcommand{\ExtensionTok}[1]{#1}
\newcommand{\FloatTok}[1]{\textcolor[rgb]{0.00,0.00,0.81}{#1}}
\newcommand{\FunctionTok}[1]{\textcolor[rgb]{0.00,0.00,0.00}{#1}}
\newcommand{\ImportTok}[1]{#1}
\newcommand{\InformationTok}[1]{\textcolor[rgb]{0.56,0.35,0.01}{\textbf{\textit{#1}}}}
\newcommand{\KeywordTok}[1]{\textcolor[rgb]{0.13,0.29,0.53}{\textbf{#1}}}
\newcommand{\NormalTok}[1]{#1}
\newcommand{\OperatorTok}[1]{\textcolor[rgb]{0.81,0.36,0.00}{\textbf{#1}}}
\newcommand{\OtherTok}[1]{\textcolor[rgb]{0.56,0.35,0.01}{#1}}
\newcommand{\PreprocessorTok}[1]{\textcolor[rgb]{0.56,0.35,0.01}{\textit{#1}}}
\newcommand{\RegionMarkerTok}[1]{#1}
\newcommand{\SpecialCharTok}[1]{\textcolor[rgb]{0.00,0.00,0.00}{#1}}
\newcommand{\SpecialStringTok}[1]{\textcolor[rgb]{0.31,0.60,0.02}{#1}}
\newcommand{\StringTok}[1]{\textcolor[rgb]{0.31,0.60,0.02}{#1}}
\newcommand{\VariableTok}[1]{\textcolor[rgb]{0.00,0.00,0.00}{#1}}
\newcommand{\VerbatimStringTok}[1]{\textcolor[rgb]{0.31,0.60,0.02}{#1}}
\newcommand{\WarningTok}[1]{\textcolor[rgb]{0.56,0.35,0.01}{\textbf{\textit{#1}}}}
\usepackage{graphicx,grffile}
\makeatletter
\def\maxwidth{\ifdim\Gin@nat@width>\linewidth\linewidth\else\Gin@nat@width\fi}
\def\maxheight{\ifdim\Gin@nat@height>\textheight\textheight\else\Gin@nat@height\fi}
\makeatother
% Scale images if necessary, so that they will not overflow the page
% margins by default, and it is still possible to overwrite the defaults
% using explicit options in \includegraphics[width, height, ...]{}
\setkeys{Gin}{width=\maxwidth,height=\maxheight,keepaspectratio}
\IfFileExists{parskip.sty}{%
\usepackage{parskip}
}{% else
\setlength{\parindent}{0pt}
\setlength{\parskip}{6pt plus 2pt minus 1pt}
}
\setlength{\emergencystretch}{3em}  % prevent overfull lines
\providecommand{\tightlist}{%
  \setlength{\itemsep}{0pt}\setlength{\parskip}{0pt}}
\setcounter{secnumdepth}{0}
% Redefines (sub)paragraphs to behave more like sections
\ifx\paragraph\undefined\else
\let\oldparagraph\paragraph
\renewcommand{\paragraph}[1]{\oldparagraph{#1}\mbox{}}
\fi
\ifx\subparagraph\undefined\else
\let\oldsubparagraph\subparagraph
\renewcommand{\subparagraph}[1]{\oldsubparagraph{#1}\mbox{}}
\fi

%%% Use protect on footnotes to avoid problems with footnotes in titles
\let\rmarkdownfootnote\footnote%
\def\footnote{\protect\rmarkdownfootnote}

%%% Change title format to be more compact
\usepackage{titling}

% Create subtitle command for use in maketitle
\providecommand{\subtitle}[1]{
  \posttitle{
    \begin{center}\large#1\end{center}
    }
}

\setlength{\droptitle}{-2em}

  \title{notes}
    \pretitle{\vspace{\droptitle}\centering\huge}
  \posttitle{\par}
    \author{Nate Lant}
    \preauthor{\centering\large\emph}
  \postauthor{\par}
      \predate{\centering\large\emph}
  \postdate{\par}
    \date{1/14/2020}


\begin{document}
\maketitle

Just looking at the introductions\ldots{}

\hypertarget{tncs-and-disabled-access-sf}{%
\subsection{TNC's and Disabled Access
(SF)}\label{tncs-and-disabled-access-sf}}

Since establishing oversight of TNCs in 2013, the California Public
Utilities Commission (CPUC) has promulgated only a few regulations and
minimal oversight to ensure equal access for passengers with
disabilities. New legislation, effective January 1, 2019, known as
Senate Bill 1376: The TNC Access for All Act (Hill), provides the CPUC
with the mandate to improve access to TNC service for wheelchair users
and others with disabilities, as well as the opportunity to work with
stakeholders to build public trust and increase transparency.

San Francisco is not the only large, urban city addressing accessibility
of TNCs. A review of Boston, Chicago, and New York City found that peer
cities are grappling with similar challenges and opportunities to
improve access to TNCs for persons with disabilities.

Riders with disabilities, like the general public, want to have choices.
For example, bus service may work well for a disabled person's trips to
work and school, but they may want to use a taxi or TNC on an evening
after a movie. Riders also want to choose whether to pay less by sharing
a ride or to spend more to go directly to their destination.

The arrival of emerging mobility services has expanded transportation
options for some but it has not expanded options equally for all. For
approximately 90,000 San Francisco residents with disabilities (almost
11\% of the population) TNCs may not be an option either some or all the
time. While people with disabilities are more reliant on for-hire
services and make twice the number of for-hire trips than non- disabled
persons per year, they are more reliant on taxicabs. People with
disabilities report taking twice as many taxi trips as TNC trips
{[}@Schaller2018{]} while overall there are approximately 12 times as
many TNC trips as taxi trips during a typical weekday in San Francisco
(San Francisco County Transportation Authority, June 2017).

The rapid expansion of TNC services has also degraded the quality and
availability of on-demand transportation access for riders who require a
wheelchair accessible vehicle by upending the existing taxi industry.
The subsequent reduction in accessible ramp taxis has compromised the
availability of accessible taxis under the San Francisco Paratransit
Taxi and Paratransit Plus programs {[}@Consulting2018{]}.

\hypertarget{ny-state-tnc-accessibility-task-force}{%
\subsection{NY State TNC Accessibility Task
Force}\label{ny-state-tnc-accessibility-task-force}}

There is a legislative history (if I find that relevant)

In general, the cost of para-transit is high for many communities. For
instance, ``The average cost of operating a single paratransit trip is
about \$23 in the U.S., compared with less than \$4 for the average trip
on bus or light rail. In Boston, the average cost per ride is about
\$45, in Washington, about \$50, and in New York, nearly \$57. Transit
agencies nationwide logged about 223 million paratransit trips at a cost
exceeding \$5.1 billion --- about 12 percent of total transit operating
costs --- in 2013, according to the most recent data from the American
Public Transportation Association.
(\url{https://www.sandiegouniontribune.com/sdut-transit-systems-eye-uber-lyft-for-savings-on-the-2016apr10-story.html}).

In 2018, the Taxi and Limousine Commission (TLC) in New York City issued
a mandate requiring Uber, Lyft and Via to make wheelchair accessible
service a growing part of their operations. While this particular
mandate was not adopted, a settlement was reached in the New York State
Supreme Court. The NYC TLC retained the mandate that would require TNCs
to meet a wait-time requirement. The wait time requirement states that,
by 2021, TNCs must either service at least 80 percent of requests for
wheelchair-accessible vehicles in under 10 minutes and 90 percent in
under 15 minutes, or associate with a company that has the capacity to
meet those requirements {[}@Report2019{]}.

\hypertarget{cases-mentioned}{%
\paragraph{Cases mentioned}\label{cases-mentioned}}

Perhaps more significantly, the Massachusetts Bay Transportation
Authority, serving the greater Boston area, instituted a policy whereby
individuals with disabilities can have their rides subsidized with
riders only paying the first \$2, and MBTA covering the next \$13.35
This is not only a significant benefit for riders with disabilities, it
is also a benefit to MBTA. Boston's door-to-door service for riders with
disabilities and elderly (the Ride) has an annual budget of over \$100
million a year.36 Under ``The Ride'' program each ride costs \$31, but
with the partnership with Uber and Lyft, these rides will cost the
agency \$13 -- a 70\% savings.

In Washington D.C., Metro launched the Abilities Ride program in
partnership with Uber and Lyft.37 In this program, the individual pays
the first \$5, Metro pays the next \$15, and then the individual is
responsible for any amount over \$20.38 This allows riders to take up to
4 rides per day, get same day services, and be accompanied by one
personal care assistant at no extra cost.39 However, WMATA was
criticized: ``The rideshare option is expected to be popular among
customers who don't need wheelchair-accessible vehicles to travel;
Still, some people with disabilities and advocates have been critical of
Metro's intent to partner with the companies, saying they lack
wheelchair-accessible vehicles and training in dealing with special-
needs populations.''40

In other communities, like Chicago, the presence of UberWAV (Wheelchair
Accessible Vehicles) appears to be expanding the accessibility of TNCs
to individuals who utilize a wheelchair. In fact, the Chicago Tribune
reported in late 2017 that Uber had ``65 wheelchair-accessible vehicles
on the road available through the app.''41 There was clearly more work
to be done, and Uber's Chicago General Manager acknowledges that,
``developing and implementing new solutions to the mobility challenge
faced by the disability community is an issue we take very
seriously.''42

\hypertarget{travel-patterns-of-american-adults-with-disabilities}{%
\subsection{Travel Patterns of American Adults with
Disabilities}\label{travel-patterns-of-american-adults-with-disabilities}}

An estimated 25.5 million Americans have disabilities that make
traveling outside the home difficult. They accounted for 8.5 percent of
the population age 5 and older in 2017. An estimated 13.4 million of
these Americans---more than half---are adults age 18 to 64, the age
group with typically high labor force participation. 4.3 million
Americans use some kind of wheelchair (manual, mechanical or an electric
scooter). The NHTS also does not include people living in nursing homes
or other group quarters.

\begin{Shaded}
\begin{Highlighting}[]
\NormalTok{nhts_persons }\OperatorTok\StringTok{ }
\StringTok{  }\KeywordTok{filter}\NormalTok{(r_age }\OperatorTok{>}\StringTok{ }\DecValTok{5}\NormalTok{) }\OperatorTok
\StringTok{  }\KeywordTok{mutate}\NormalTok{(}\DataTypeTok{test.ability =} 
           \KeywordTok{case_when}\NormalTok{(}
\NormalTok{               w_chair }\OperatorTok{==}\StringTok{ "07"} \OperatorTok{|}\StringTok{ }\NormalTok{w_mtrchr }\OperatorTok{==}\StringTok{ "08"} \OperatorTok{|}\StringTok{ }\NormalTok{w_scootr }\OperatorTok{==}\StringTok{ "06"} \OperatorTok{~}\StringTok{ "Wheelchair"}\NormalTok{,}
\NormalTok{               medcond }\OperatorTok{==}\StringTok{ "01"} \OperatorTok{~}\StringTok{ "Disabled"}\NormalTok{,  }
\NormalTok{               medcond }\OperatorTok{==}\StringTok{ "02"} \OperatorTok{~}\StringTok{ "Abled"}\NormalTok{)}
\NormalTok{           ) }\OperatorTok\StringTok{ }
\StringTok{  }
\StringTok{  }\KeywordTok{group_by}\NormalTok{(test.ability) }\OperatorTok
\StringTok{  }\KeywordTok{summarise}\NormalTok{(}\DataTypeTok{Survey =} \KeywordTok{n}\NormalTok{(),}
            \DataTypeTok{population =} \KeywordTok{sum}\NormalTok{(wtperfin)) }\OperatorTok
\StringTok{  }\KeywordTok{mutate}\NormalTok{(}\DataTypeTok{Population =}\NormalTok{ population,}
         \StringTok{`}\DataTypeTok{Distribution(%)}\StringTok{`}\NormalTok{ =}\StringTok{ }
\StringTok{           }\KeywordTok{percent}\NormalTok{(population}\OperatorTok{/}\KeywordTok{sum}\NormalTok{(population), }\DataTypeTok{accuracy =} \FloatTok{0.1}\NormalTok{)) }\OperatorTok
\StringTok{  }\KeywordTok{select}\NormalTok{(}\OperatorTok{-}\NormalTok{population) }
\end{Highlighting}
\end{Shaded}

\begin{verbatim}
## # A tibble: 4 x 4
##   test.ability Survey Population `Distribution(%)`
##   <chr>         <int>      <dbl> <chr>            
## 1 Abled        236051 271384034. 91.4%            
## 2 Disabled      20801  21048869. 7.1%             
## 3 Wheelchair     4335   4354116. 1.5%             
## 4 <NA>            128    127248. 0.0%
\end{verbatim}

\hypertarget{intelligent-paratransit}{%
\subsection{Intelligent Paratransit}\label{intelligent-paratransit}}

Residents of cities who are physically unable to use public
transportation, including the disabled and mobility- impaired elderly,
are offered car or van rides by paratransit services. Required by an
unfunded 1990 Americans with Disabilities Act mandate (https:
//www.law.cornell.edu/cfr/text /49/37.21), paratransit systems are
enormous: in New York City, paratransit serves 144,000 subscribers at
\$456 million per year; in the Chicago region, 50,000 subscribers are
served at \$137 million per year; in Boston, 80,000 at \$75 million per
year. These operations grow annually with new registrations and costs.
Furthermore, their rides are reportedly poor experiences ().

\hypertarget{impacts-of-limited-acces-bascom-and-christensen}{%
\subsection{Impacts of limited acces (Bascom and
Christensen)}\label{impacts-of-limited-acces-bascom-and-christensen}}

In order for individuals to obtain employment, goods and services,
healthcare, education, and interact socially, access to trans- portation
is critical (Cassas, 2007; Preston and Raje, 2008) For example, a lack
of access to transportation not only limits access to employment
opportunities, but can also make it more difficult to find employment by
limiting access to employment center and interview locations (Kenyon et
al., 2002; Department of Environment Transport and the Regions, 2000).
Similarly, healthcare and education are often not equally distributed in
a community, making access difficult for individuals who do not live
near these services (Martens, 2012)

\hypertarget{strategic-plan-2019-2022}{%
\subsection{Strategic Plan 2019-2022}\label{strategic-plan-2019-2022}}

Personal mobility is essential to the success of America's citizens,
communities, and economy. Transportation enables mobility by connecting
individuals to their homes, jobs, and communities. Despite its
significance, millions of Americans lack access to reliable
transportation due to disability, income, or age. Inadequate
transportation limits the mobility of these individuals and prevents
them from accessing jobs, medical care, healthy food, education, social
services, and other community activities.

\hypertarget{policies-and-practices-for-meeting-ada-paratransit-demand}{%
\subsection{Policies and Practices for Meeting ADA Paratransit
Demand}\label{policies-and-practices-for-meeting-ada-paratransit-demand}}

Although paratransit ridership is slightly more than 1\% of the total
transit ridership, paratransit costs comprised 9\% of transit operating
costs; therefore, efficiencies are needed to address the ever-
increasing costs of meeting the civil rights requirements of the
Americans with Disabilities Act (ADA) for paratransit service. From
1992---the first year of ADA-complementary paratransit service---to
2004, paratransit ridership in the United States increased by 58.3\%, to
more than 114 million trips, most of which were ADA-complementary
paratransit trips. In addition, the operating cost per trip for
paratransit service was \$22.14, whereas for all other modes, the
operating cost per trip was \$2.75 (per trip costs calculated from APTA
data).

Efficiencies are needed to address the ever-increasing cost of meeting
the civil rights requirements of the Americans with Disabilities Act
(ADA) for paratransit service. An underlying purpose of the ADA is to
provide equal opportunity, full participation, and independence to
persons with disabilities. Transit

\hypertarget{schaller-new-automobility}{%
\subsection{Schaller: New
Automobility}\label{schaller-new-automobility}}

People with disabilities are more reliant on for-hire services, in
particular taxicabs, than non-disabled persons. While nondisabled people
make 4.1 for-hire trips annually, people with disabilities make twice as
many trips (8.2 per year). (National data only; sample size too small
for geographic detail.) People with disabilities are also more reliant
on taxicabs than the general population. People with disabilities take
5.9 taxi trips annually, twice their use of TNCs (2.3 trips per year).

There is a long history of taxicabs participating in Dial-A-Ride
programs for seniors and persons with disabilities who lack access to a
personal car or the financial means to pay for a taxi. Public subsidies
are needed for patrons to obtain medical care, go shopping, socialize at
senior centers, attend religious services and so forth.

\hypertarget{case-studies}{%
\paragraph{Case Studies}\label{case-studies}}

Laguna Beach, for example, contracted with Uber to supplement
transportation for senior and disabled passengers following curtailments
of local bus service.

The Pinellas Suncoast Transit Authority in the Tampa and St.~Petersburg,
Florida area, conducted a two-year pilot with Uber, a cab company and a
wheelchair van provider for on-demand trips at night to or from work to
participants in an agency program for transportation-disadvantaged
persons.

After an initial microtransit pilot involving the now-defunct company
Bridj, the Kansas City Area Transportation Authority is using taxis in
its RideKC Freedom program, serving older adults and persons with
disabilities with same-day service scheduled through a mobile app or by
telephoning a call center.

Via is developing with the city of Berlin, Germany a van service that
complements existing transit service, focusing on late night and weekend
travel.

TNCs have recently started to participate in programs that supplement
ADA paratransit. A prime example is the pilot by the Boston area transit
agency (MBTA) that involves Uber, Lyft and other companies. ADA
paratransit users are offered the option of using one of these three
companies instead of the regular ADA service. They can make same-day
reservations instead of having to call a day or more in advance. Riders
pay the same \$2 fare and any amount over \$15 (making for a subsidy of
up to \$13 per trip). Lyft provides a call center under its Lyft
Concierge program, while Uber addressed smartphone issues by giving away
smartphones to some users.

Another example is the transit agency in Las Vegas, Nevada, which began
a pilot earlier this year with Lyft to provide on-demand paratransit
service.

\hypertarget{use-of-taxis-in-public-transportation-for-people-with-disabilities-and-older-adults-2016}{%
\subsection{Use of Taxis in Public Transportation for People with
Disabilities and Older Adults
(2016)}\label{use-of-taxis-in-public-transportation-for-people-with-disabilities-and-older-adults-2016}}

Many transit agencies use taxis as part of their required ADA
paratransit service and to provide a same-day service that is not a
formal part of ADA paratransit service.


\end{document}
